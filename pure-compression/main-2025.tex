\documentclass[12pt]{article}

\usepackage{preamble}

% document specific

\newcommand{\anote}[1]{\textcolor{red}{\small {\textbf{(Ashwin:} #1\textbf{) }}}}

\title{
\textbf{ABC }~\thanks{Any ack common to all authors
}
}

\author{
ABC XYZ~\thanks{affiliation details, address, email, etc.
} \\
Affiliation
\and
BCD WXY~\thanks{abc
} \\
U.\ Waterloo
}

\date{???}

\begin{document}

\maketitle

\begin{abstract}
TO BE COMPLETED
\end{abstract}

\section{Introduction}
\label{sec-introduction}

\paragraph{Acknowledgements.}
Personal acks, grants, etc.

\section{Section 2}
\label{sec-2}

\subsection{Pure State Source}

% TODO: Problem setup?
\subsubsection{Problem Setup}
% describe reduction to unitary case, norm calculation, etc

Given an ensemble $\{p_x, \ket{\rho_x}\}$, consider the diagonalization
\[\sum_x p_x\ketbra{\rho_x}{\rho_x} = \sum_x \beta_x \ketbra{\psi_x}{\psi_x}\]
Where the collection $\{\psi_x\}_x$ is pairwise orthogonal.
Without loss of generality we may assume that the above density matrix has full support,
as by taking a projection onto its support we find a trivial compression of the ensemble into a lower dimension with zero error. 
Setting $n \eqdef \dim(A)$, we may then index the above sum ranging from $\{1, ..., n\}$.

Up to a unitary on the reference and up to the reindexing above, we may then write the purification of the ensemble as:
\begin{equation}\label{eq:purif}
    \ket{\rho}^{RA} \eqdef \sum_{i=1}^n \sqrt{\beta_i}\ket{\psi_i}^A \ket{i}^R
\end{equation}
Given a target space $A'$ with $\dim(A') < \dim(A)$, we wish to find the following maximization:

% substack is a bit misaligned, can move upwards maybe
\[\max_{
    \substack{U^{AC_1 \to A'C_1'} \text{unitary} \\ V^{A'C_2 \to AC_2'} \text{unitary}}}
    \norm{
        \left(\bra{\rho}^{RA} \otimes \one^{C_1'C_2'}\right)
        VU\ket{\rho^{RA}}\ket{\bar{0}}^{C_1C_2}
    }_2^2
\]
\subsubsection{Upper Bounds}
With respect to the purification in \cref{eq:purif},
we first characterize the action of $U$ on the subspace spanned by 
$\{\ket{\psi_x}^A\ket{\bar{0}}^{C_1}\}_x$.
Notice that the action of $U$ elsewhere is lost in the maximization, 
it suffices to retain that the image of an orthonormal set remains orthonormal.

If $d \eqdef \dim(A')$, we may let $\{\ket{j}\}_{j=1}^d$ be an ONB for $A'$.
We may then decompose each $U\ket{\psi_i}^A\ket{\bar{0}}^{C_1}$ as:
\begin{equation}\label{eq:decomp}
    U\ket{\psi_i}\ket{\bar{0}} = \sum_j \lambda_{ij}\ket{j}^{A'}\ket{\xi_{ij}}^{C_1'}
\end{equation}
Note that $\{\ket{\psi_x}\}_x$ is pairwise orthogonal, and $U$ is unitary.
This will be neccessary later.

Next, the analysis of action of $V$ similarly can be restricted to only on the span of
$\{\ket{j}^{A'}\ket{\bar{0}}^{C_2}\}$
which is once again orthonormal.
Designate:
\[V\ket{j}^{A'}\ket{\bar{0}}^{C_2} = \ket{\sigma_j}^{AC_2'}\]
The combined action of $VU$ on $\ket{\rho}^{RA}\ket{\bar{0}}^{C_1C_2}$ then is as follows:
\[VU\ket{\rho}^{RA}\ket{\bar{0}}^{C_1C_2}
    = \sum_{i=1}^n \sqrt{\beta_i} \ket{i}^R
        \left(\sum_{j=1}^d \lambda_{ij}\ket{\sigma_j}^{AC_2'}\ket{\xi_{ij}}^{C_1'}\right)
\]
Expanding out $\bra{\rho}^{RA}\otimes \one^{C_1'C_2'}$
with respect to \cref{eq:purif} and contracting the inner product on $R$ first to simplify,
we obtain the following norm calculation:
\[\max \norm{\sum_{ij} \beta_i \lambda_{ij} 
    \left(\bra{\psi_i}^A \otimes \one^{C_2'}\right)\ket{\sigma_j}^{AC_2'}
    \ket{\xi_{ij}}^{C_1'}}_2^2
\]
We simplify this maximization by dropping the square due to monotonicity of squaring,
and apply Cauchy-Schwarz to upper bound the above norm (without the square) by:
\[\sum_i \beta_i \norm{
    \left(\bra{\psi_i}^A \otimes \one^{C_2'}\right)
    \left(\sum_j \lambda_{ij}\ket{\sigma_j}\ket{\xi_{ij}}\right)}_2\]
Due to the orthonormal conditions imposed on $\ket{\sigma_j}$ and $\lambda_{ij}, \ket{\xi_{ij}}$,
we in fact have that
\begin{equation}\label{eq:def_varphi}
    \ket{\varphi_i} \eqdef \sum_j \lambda_{ij}\ket{\sigma_j}^{AC_1'}\ket{\xi_{ij}^{C_2'}}
\end{equation}
Is indeed a unit vector.
Letting
\[\alpha_i \eqdef \norm{
    \left(\bra{\psi_i}^A \otimes \one^{C_2'}\right)
    \ket{\varphi_i}}_2\]
And applying tracial properties, we may calculate that:
\begin{align*}
    \alpha_i^2 &= \trace\left((\ketbra{\psi_i}{\psi_i}^A\otimes \one^{C_1'C_2'})\ketbra{\varphi_i}{\varphi_i}\right)\\
    &= \trace\left((\ketbra{\psi_i}{\psi_i}^A \otimes \one^{C_1'})\trace_{C_2'}\ketbra{\varphi_i}{\varphi_i}\right)
\end{align*}
But it clear from \cref{eq:def_varphi} that for every $i$ we have that
$\trace_{C_2'}\ketbra{\varphi_i}{\varphi_i}$
has both support and range contained in the span of
$\{\ket{\sigma_j}\}$.
Letting $P$ denote the projection onto this span
(and noting that by orthogonality that $P$ is of rank $d$ exactly)
we have that
$\trace_{C_2'}\ketbra{\varphi_i}{\varphi_i} \preceq P$
since the left cannot have eigenvalues larger than 1.

It follows that we may write:
\begin{align}
    \sum_i \alpha_i^2 
        &\le \sum_i \trace\left(
            \ketbra{\psi_i}{\psi_i}^A\otimes \one^{C_1'}
            P \right) \\
    & = \trace(\one^{AC_1'}P) = d
\end{align}
Where the second last equality follows from $\psi_i$ being a basis for $A$ by assumption.

We have then reduced the above maximization problem to the following:
\begin{equation*}
\begin{array}{rl}
    \text{maximize}    & \sum_i \beta_i \alpha_i \\
    \text{subject to:} & \sum_i \alpha_i^2 \le d
\end{array}
\end{equation*}

\bibliographystyle{plain}
\bibliography{references}

% \appendix

% \include{appendix}

\end{document}

